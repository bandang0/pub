
\part{Perspectives on the study of binary neutron star mergers}

MMT170817 is the first neutron star merger event observed, and is was so with the capabilities of multi-messenger astronomy. As we have shown, it is a peculiar event in several respects, and though its afterglow informs us on some aspects of the dynamics which occurred in the event an on the possibility that a jet was produced, many questions remain. What is to come in the study of binary neutron star mergers?

\section{Observation perspectives}
\subsection{Future rates of events and counterparts}

A single event was observed until now. As we have shown, the overall observations are the product of various factors, some intrinsic to the event --~the ejection energy, velocity, mass, etc.~--, some intrinsic to the event locus --~external medium density and microphysics parameters~--, and finally some extrinsic --~viewing angle, distance. We may ask what is the diversity of these events, and can the external diversity be \it{factored out} in order to obtain a description of the merger event itself?

Clearly, any result in this sense will come from a large number of observations. We will review here some estimates for future rates of mergers and electromagnetic counterparts.

\bf{Gravitational wave signals.} The detection of GW170817 allows to infer to BNS merger rate in the local Universe to be $1540^{+3200}_{-1220}$~Gpc\sp{--3}yr\sp{--1} \citep{37}. The next joint three-detector observation run named O3 should start mid 2019, as announced by the LSC-Virgo Collaboration \citep{54}. It should last one year and the BNS range\footnote{The sky-position averaged distance at which the GW from a merger of a 1.4$\Ms$-1.4$\Ms$ binary neutron star can be detected by a detector. It takes into account the instrumental sensitivity variations due to the orientation of the interferometer with respect to the source. Another frequently cited figure is the BNS \it{horizon}, which is the distance to which an  optimally oriented such binary merger can be detected. The range is 2.26 times lower than the horizon.} is predicted to $\sim$~150~Mpc. Taking a three-instrument duty cycle (the fraction of observation time when all three instruments are online) of 45\% \citep{54}, this results in $10^{+20}_{-8}$ BNS merger GW detections during the O3 run. As we have already described, the detection range is determined by the largest range of the three instruments, while the ranges of the others impact the localization capabilities.
At the design sensitivity of 190~Mpc, predicted for 2021, this is $20^{+40}_{-16}$ per accumulated year of observing run.

\bf{Gravitational waves with kilonova. }What are the chances to detect a kilonova associated to a GW signal? The first remark is that kilonovae emissions are likely isotropic, as the typical scenario suggests. Thus, contrarily to GRBs, the only limiting factor to kilonovae detection is the distance to the source, and not the observing angle.

The second remark is that in the case of binary neutron star mergers, nuclear processes are likely to consistently occur (and drive a kilonovae emission) in the sequel of merger events. These will happen either in dynamical ejecta in the case of collisions, in the tidal ejecta in case of a tidal disruption event (for highly asymmetrical binaries) or in the wind produced by some accretion disk formed of these ejectas. Thus, a kilonova emission (possibly with various components due to different ejecta of varying opacities and masses) should be associated to any BNS merger event.

Finally, it seems that the only parameters relevant to determining the composition and the dynamics of the kilonova are the initial masses and spins of the neutron stars. Indeed, the final orbits of the binary are fully determined by these, and thus the mass and velocity of the ejecta is determined solely by these, and its composition and opacity are determined only by the (unique albeit unknown) equation of state of nuclear matter. Therefore, we predict little variability in the kilonovae of binary neutron star mergers\footnote{Also note that if the kilonova is indeed formed of two components (accretion disk and wind, as discussed in Section \ref{kilonova}), then the wind component which is likely oriented along the angular momentum of the binary, could be suppressed by large viewing angles.}. If kilonovae indeed all have the same absolute magnitude of --16 like AT~2017gfo, then kilonovae can be detected out to $\sim$~250~Mpc, assuming a photometric quality maximum-detectable magnitude of 21.

Thus, the kilonova range is larger than the BNS GW range, even for dimmer kilonovae. Given that GW are systematically the trigger for the detection of merger events, and that they are the only source of a restrictive-enough error box for the search of electromagnetic counterparts, we conclude that the kilonova rate should be close to the BNS merger GW rate.

\bf{Afterglow detections. }Contrarily to kilonovae, the afterglow emission depends on many parameters, as we have detailed in this work. There will likely be a large diversity in afterglow emissions. Nonetheless, once a transient optical source is detected in the error box provided by the GW signal, then the source is pinpointed to sub-arcsecond precision, and an afterglow can be either detected, or constrained by upper limits. Thus the rate of astronomically-interesting study cases concerning BNS merger afterglows is the same as the kilonova rate, which in turn amounts to the GW detection rate.


\bf{Coincident GRBs. }Unlike kilonovae and surely afterglows, our current understanding of GRBs is that they are collimated emissions, and thus are suppressed by large viewing angles. What's more, the gravitational interferometer signal has a strong and non-trivial dependence (the so-called \it{antenna pattern}) on the direction of the GW wave-vector relative to the detector, and on the polarization of the wave. Therefore, estimating the rates of detection of GRBs coincident with GW events is delicate. This is all the more true that this rate strongly depends on the strength of the GRB, which in turn is subject to the dissipation mechanism at the origin of the GRB, which as detailed earlier remains unclear in the case of GRB170817A.

A full-fledged calculation of coincident GRB and GW detection rates taking all of these effects into account must be done, and any estimate without this calculation would be hazardous.

In conclusion, we can say that apart from the GRB counterpart, for which it is difficult to make rate predictions, the rates of both BNS merger-induced kilonova detections and remnant detections or upper limits is essentially the same as the rate of BNS merger GW detections. The latter is conditioned by the population of binary neutron systems in the local Universe, and by the sensitivity of the gravitational interferometers. A current prediction is $10^{+20}_{-8}$ detections for the one-year O3 run to come.

\subsection{Insights on other astronomical observables}

In this work we were concerned with the afterglow light curve of MMT170817. As we have seen, it allows us to infer much information on the global phenomenon of neutron star mergers: from the exterior medium to the structure of the ejectas and the possibilities for a relativistic jet.

What may other astronomical observables teach us on this phenomenon?

\bf{Polarization of the afterglow emission.} The polarization of the afterglow emission can be calculated similarly to the afterglow flux. Synchrotron emission is naturally polarized, and the overall polarization of the afterglow will depend on geometrical hypotheses concerning the shock itself (jet-like, spherical) and the magnetic field. For example, spherical symmetry arguments allow to conclude that the linear polarization of the emission from  a spherical shock front vanishes. On the other hand, if the remnant is jetted (in a cone shape), then the polarization can be important, if the system is seen off-axis. Also, according to whether the magnetic field is contained in the shock plane or has a non-zero longitudinal component, the linear polarization of the emission will change.

We notice here that if the ejecta is composed of various ejectas with different geometries (such as a jet alongside a quasi-spherical outflow), it is possible that a jetted ejecta may not be visible in photometry, due to its small flux compared to the quasi-spherical outflow, but readily discernable by the polarization that it induces in the overall flux. Certain authors such as \citet{7} predict linear polarizations as large as 60\% in some models for MMT170817 where a jetted ejecta formed along with a quasi-spherical ejecta.

At these fluxes ($\sim$~10 $\mu$Jy), the polarization of the emission is readily measured, and can be a supplementary means of sounding the presence of a jet through the afterglow emission. An upper limit on the linear polarization of the afterglow of MMT170817 has very recently been measured \citep{63}. The interpretation of this observation will teach us more on the merger event, and the modelization of the polarization of afterglows is already underway in order to respond to observations of this kind in the future.

\bf{Radio imaging of the remnant.} Given the relative proximity of the locus of MMT170817 and the typical velocities of the remnant's expansion, the angular diameter of the remnant can be estimated to be $\sim$~0.1~--~10~mas at the time of peak flux \citep{7}. This angular resolution is within the capabilities of global very long base radio interferometry networks. Then, the actual shape of the remnant would be accessible, and the geometry and dynamics of the outflow could be assessed in better detail.

\subsection{Insight on future modelization}
\label{modelization}
Here we will describe some possibilities for future modelization of the afterglows of neutron star mergers. In addition to the polarization and imaging of the remnant we have just discussed, we may study the reverse shock of the remnant as well as a possible angular structure for the outflow.

\bf{Reverse shock.} Observations of supernovae remnants \citep[see][]{56} attest the presence of a \it{reverse shock}, as in figure \ref{schema}. This structure is also discussed in the context of GRB afterglows. It is the interface between the catching-up ejecta and the accumulated ejecta which has already caught up. At this shock, it is the ejecta (and not the interstellar material) which dissipates its internal energy by radiating.

In this work, we have explained and fit the afterglow radio data with emission from the front shock only. The reverse shock may also contribute synchrotron radiation to the afterglow, and this remains to be explored. Reverse shock structures can be inhibited by magnetized ejectas. Thus, if it is found that the contribution of the reverse shock does not provide a better understanding of the radio data, it can mean that the reverse shock did not form in the case of MMT170817, and that the ejecta is highly magnetized. This can be a first insight on the nature and magnetic state of the ejecta from the merger event, and is the following step we will take in our work on MMT170817.


\bf{Angular-structured outflow.} We have considered mono-kinetic ejectas, and radially-structured ejectas. Nonetheless, an outflow with an angular structure is also a possibility. That is, an outflow with a central energetic component, reminiscent of a jet structure, and slower --~though relativistic~-- lateral wings. This for example is the dynamical structure of a \it{choked jet} in cocoon models \citep{42, 5}. These models introduce much degeneracy because a continuous distribution of Lorentz factors $\Gamma(\theta)$ must be parametrized. Furthermore, these structures break the spherical symmetry which renders $F^{\rm{iso}}$ readily calculable, and these models thus require the complete integration of flux as in Eq.~\ref{in}.

\section{Open questions}

\bf{What is the origin of GRB170817A? }The non observation of a jet-induced afterglow has led us to constrain the opening angle and kinetic energy of any relativistic jet which would have been produced during the event. If any jet is so constrained, can it still be at the origin of the GRB? Among the models for GRB production through internal shocks, magnetic reconnection, etc., which allow for a GRB as observed by jets as constrained? If the answer is none, then another mechanism for the production of GRBs specifically for neutron star mergers must be found.

In this case, if the GRB is produced not by a jet but by a wider outflow, then \it{(i)} the GRB might not be as suppressed by larger viewing angles as thought, and the rates for coincident GW and GRB detections could be revised to the high side, and \it{(ii)} this wider outflow may be the early-state of an outflow responsible for the afterglow, and the GRB phase may be more readily studied from the angle of the afterglow.

\bf{What is the nature of the resulting compact object? }The GW data is inconclusive regarding the nature of the final central object. Can this object's nature be resolved studying the electromagnetic counterparts?

First off, we can predict that the total ejected mass varies according to whether a black hole is immediately formed after the merger, or whether a compact object producing a lesser-constraining gravitational field (such as a neutron star) exist for some time. The mass of these ejectas being inferred by kilonova and afterglow studies, this constitutes a first approach to resolving the nature of the remaining object.

Secondly, if there are indeed two components to the kilonova (red and blue, as discussed in Section \ref{kilonova}), we can expect that the wind-driven blue kilonova which is raised above the accretion disk at times later than the red kilonova will be influenced by the local magnetic and gravitational conditions. In particular, a magnetar (and thus a neutron star central object) can help to extract angular momentum from the disk and drive the wind and thus the blue kilonova. Similarly, we predict that late-time activity of the central engine, such as X-ray flares, can be enhanced or inhibited by strong magnetic conditions. For MMT170817, the observation of a X-ray flare has been claimed, though the significance of this event is still unclear, and in any event the link between such late activiy and the nature of the central object will only be established by observing a large number of merger events.


\bf{What is the intrinsic diversity of the merger phenomenon? }As we have detailed earlier, the observation of many events in the multi-messenger context will allow us to probe the diversity (intrinsic and extrinsic) of the event as a whole, and of its components --~kilonova, GRB, afterglow. This diversity is strongly correlated to the diversity of the binaries themselves and of the host environment, and likely conditions the detection rates of these events.\\

All of these questions pertain to a new domain of astrophysics --~binary compact object mergers~-- for which the time of observations has just begun, and I hope to contribute positively in the future to the understanding of this fascinating phenomenon.
