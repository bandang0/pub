
\part{Perspectives on the electromagnetic counterparts of binary neutron star mergers}

\section{Future rates of events and counterparts}

\section{Insights on other astronomical observables}

In this work we were concerned with the afterglow light curve of MMT170817. As we have seen, it allows us to infer much information on the global phenomenon of neutron star mergers: from the exterior medium to the structure of the ejectas and the possibilities for a relativistic jet. 

What may other astronomical observables teach us on this phenomenon?

\bf{Radio imaging of the remnant.}

\bf{Polarisation of the afterglow emission.} Considerations similar to those which result in the functional form of the light curve from a given distribution of memitting matter lead to 


\section{Open questions}

\bf{What GRBs can be produced by the hidden jets? }
\bf{What is the nature of the resulting compact object? }


\bf{What is the intrinsic diversity of the merger phenomenon? }