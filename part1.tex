\part{Introduction}

\section{MMT170817 -- a historical event}

\fig{gbmax}{0.5}{A nice figure}
\section{Description of the multi-messenger observations}

\subsection{GW170817: gravitational waves from the inspiral phase}

\fig{gw}{0.35}{Spectrogram of a combination of both LIGO interferometers' data from GW170817 (\cite{37}). Note the non-detection of a ring-down signal.}
\subsection{GRB170817A: the gamma-ray burst}

As illustrated in figure \ref{yonetoku}, GRB170817A is an outlier of the $E_p$-$L_{\rm{iso}}$ relation, also know as the Yonetoku relation (\cite{35}). In this relation, $E_p$  and $L_{\rm{iso}}$ are the peak photon energy and isotropic equivalent luminosity of the burst at its peak.

This observation is the first which hints to the \it{atypical} character of GRB170817A.

\fig{yonetoku}{0.35}{The Yonetoku relation for the Swift BAT 4 catalog bursts (black crosses, \cite{36}) and GRB170817A. The values of $E_p$ and $L_{\rm{iso}}$ that would have likely been measured for other viewing angles form a line in this diagram.}

Using a simple calculation to estimate the $E_p$ and $L_{\rm{iso}}$ values which would have been observed from different angles, it appears that this event is an outlier of the Yonetoku relation regardless of the viewing angle.

\subsection{AT 2017gfo: the kilonova}

\fig{kn}{0.35}{Time evolution of the spectrum of AT 2017gfo (\cite{38}).}
\subsection{The afterglow signal}

\fig{remnants}{0.35}{Some of the most long-lived afterglows from the Swift catalog (doted lines, X-ray data), and the afterglow from MMT170817.}

\section{Goal of this work and the afterglow of MMT170817 as an insight on the merger event}

The merger of a binary neutron star is a complex phenomena. It most likely implies various physical components (compact objects, jets, ejectas, winds) which are subject to many dynamical and radiative processes (shock formation, nuclear processes, synchrotron emission, etc.). A coherent description of the binary neutron star merger phenomena, from the inspiral phase to the electromagnetic afterglow, is still to be found. Most likely, the combination of all the multi-messenger observations from a lerge number of events will be necessary to obtain this accurate description, each signal bearing the signature of the state of the phenomenon at different times.

In particular, the afterglow holds information on the state of matters at late times, and is a first step one may take to approach the event in its entirety. The goal of this work is precisely to study the afterglow of MMT170817 in the perspective of later inferring information on the earlier phases of the event: the nature and formation of the resulting compact object, the origin of the gamma ray burst, the dynamical properties of the outflow, the development of the kilonova, etc.

The questions this work adresses are thus the following:

\begin{enumerate}
	\item What is the geometry and the structure of the matter responsible for the afterglow?
	\item By which means is the kinetic energy dissipated into the afterglow radiation?
	\item What are the characteristics of the medium in which this matter evolves?
	\item Was a relativitsic jet produced in the merger event?
	\item By which angle do we currently see the system?
	\item Which electromagnetic counterparts would we have seen, had we seen the event from different viewing angles?
\end{enumerate}

